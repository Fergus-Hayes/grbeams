\documentclass[onecolumn,nofootinbib]{revtex4-1}
\usepackage{graphics}
\usepackage[caption=false]{subfig}
\usepackage{amsmath}
\usepackage{hyperref}
\usepackage{amssymb}
\usepackage{color}
\usepackage{epsfig}
\usepackage{latexsym}
\usepackage{tensor}
%\usepackage{wasysym}
\usepackage{comment}
%\usepackage{graphicx}
%\usepackage{psfrag}


%\newcommand{\gw}{gravitational wave }
%\newcommand{\gws}{gravitational waves }
\newcommand{\subgw}{_{\textrm{\scriptsize{GW}}}}
\newcommand{\ee}[1]{\!\times\!10^{#1}}
\newcommand{\prob}{{\rm Pr}}
\newcommand{\grbrate}{{{\mathcal R}_{\mathrm{grb}}}}
\newcommand{\cbcrate}{{{\mathcal R}}}
\newcommand{\diff}{{\mathrm d}}
\newcommand{\rhostar}{{\rho^*}}

\def\imbh#1{intermediate mass black hole#1(IMBH#1)\gdef\imbh{IMBH}}
\def\smbh#1{supermassive black hole#1(SMBH#1)\gdef\smbh{SMBH}}
\def\bbh#1{binary black hole#1 (BBH#1)\gdef\bbh{BBH}}
\def\bh#1{black hole#1 (BH#1)\gdef\bh{BH}}
\def\ns#1{neutron star#1 (NS#1)\gdef\ns{NS}}
\def\gw#1{gravitational wave#1 (GW#1)\gdef\gw{GW}}
\def\sn#1{core-collapse supernova#1 (CCSN#1)\gdef\sn{CCSN}}
\def\pnw#1{post-Newtonian#1 (PN#1)\gdef\pnw{PN}}
\def\eos#1{equation of state#1 (EOS#1)\gdef\eos{EOS}}
\def\grb#1{gamma-ray burst#1 (GRB#1)\gdef\grb{GRB}}
\def\amr#1{adaptive mesh refinement#1 (AMR#1)\gdef\amr{AMR}}
\def\isco#1{innermost stable circular orbit#1 (ISCO#1)\gdef\isco{ISCO}}
\def\cwb#1{Coherent WaveBurst#1 (CWB#1)\gdef\cwb{CWB}}

\newcommand{\red}[1]{{\color{red}{#1}}}
\newcommand{\about}[1]{{\color{blue}{[THIS SECTION: #1]}}}
\newcommand{\add}[1]{{\color{magenta}{[TO INCLUDE: #1]}}}
\newcommand{\JC}[1]{{\color{magenta}{[[JC #1]]}}}
\newcommand{\placeholder}[2]{{\color{red}{[PLACEHOLDER](#1):}}{\color{red}{[}}#2{\color{red}{]}}}
\providecommand{\todo}[1]{{\color{red}$\blacksquare$~\textsf{[TODO: #1]}}}
\newcommand{\mean}[1]{\langle#1\rangle}


%%%%%% Useful for draft editing
\usepackage{soul} 
\usepackage{ulem} \normalem 
\newcommand{\ec}[1]{{\noindent\color{red}{\it [[#1]] }}}
\newcommand{\laura}[1]{{\color{blue}{#1}}}
\newcommand{\LC}[1]{{\color{red}{[[LC #1]]}}}
\newcommand{\AB}[1]{{\color{blue}{[[AB #1]]}}}
\newcommand{\highlight}[1]{\colorbox{yellow}{#1}}
\newcommand{\simgt}{\mbox{$^{>}_{\sim}$}}

\begin{document}

\title{What Is $\theta$?}
\author{James Clark, Ik Siong Heng, Martin Hendry}
\date{\today}

\begin{abstract}
Some notes from James \& Siong's chat, Tuesday 28th October.  
\end{abstract}

\maketitle

\section{Mean Beaming Angle}
Let's start with the usual equation:
\begin{equation}\label{eq:rate2angle}
\grbrate=\epsilon\cbcrate(1-\cos \theta),
\end{equation}
%
where $\grbrate$ is the observed rate of sGRBs, $\cbcrate$ is the binary
coalescence rate measured from \gw{s}, $\epsilon$ is the fraction of CBCs which
result in a GRB and $\theta$ is the GRB-beaming angle.

In much of what we've spoken about to date, we've essentially assumed that
$\theta$ is constant across all GRBs.  For sGRBs in particular, that doesn't
seem to be terribly plausible.  So how do we handle the fact that $\theta$ for
GRB X is not necessarily the same as $\theta$ for GRB Y?  First, we need to
understand what $\theta$ really \emph{is}.

Equation~\ref{eq:rate2angle} relates the observed rates of GRBs and CBCs due to
beaming.  In other words, it relates the expected numbers of the respective
observations.  In the case that $\theta$ is fixed for all GRBs this is trivial.
If there is a distribution of $\theta$'s, however, the relative numbers of
events in equation~\ref{eq:rate2angle} are dictated by the expectation value for
$\theta$, which we'll call $\langle\theta\rangle$.  So, really, we should be more
explicit:
%
\begin{equation}\label{eq:rate2angle2}
    \grbrate=\epsilon \cbcrate\left\{1-\cos \left [\int_{\Theta} \theta
    p(\theta)~\diff \theta \right ]\right\}
\end{equation}
%
So, really, we're only sensitive to $\hat{\theta}$ if we only look at the raw
numbers of GRBs and CBCs.   In fact, it's pretty easy to verify this with a very
simple Monte-Carlo simulation.  E.g.,
\begin{enumerate}
    \item Generate a population of `CBC events'; this is just a sample of size
        $N_{\text{cbc}}$ of inclinations $\iota$ drawn from a uniform
        $\cos(\iota)$ distribution
    \item For each CBC event, draw a sample of a jet angle from some
        distribution with mean $\hat{\theta}$
    \item If the inclination of the CBC event is smaller than the jet angle,
        count that as a GRB
    \item The numbers you get out are $N_{\text{grb}}$ and $N_{\text{cbc}}$ 
    \item The number of GRBs one `observes' is indeed given by
        equation~\ref{eq:rate2angle2}
\end{enumerate}
%
The width of the distribution on $\theta$ only affects the numbers at the
$\sim\%$ level which is probably just sampling error.

{\bf In conclusion:}  If we start with equation~\ref{eq:rate2angle} as our
premise, then the angle of interest is just the mean of the population.

\begin{comment}
\section{Refinement by all means!}
Actually, \emph{all} of the variables in equation~\ref{eq:rate2angle} are
expectation values:
%
\begin{eqnarray}
    \mean{\grbrate} & = & \mean{\epsilon}\mean{\cbcrate}(1-\cos \mean{\theta}),
    \\
    & = & \int_{\epsilon} \int_{\cbcrate}
\cbcrate.p(\cbcrate)\left\{1-\cos\left[ \int_{\Theta} \theta.p(\theta)~\diff
\theta\right]\right\}
\end{eqnarray}
%
This also nicely encapsulates our treatment of $\epsilon$ which is really
supposed to be the probability that a given CBC results in a GRB and should be
some distribution conditioned on the progenitor physics and parameters.

I'm confused now - is this just working back towards 
\end{comment}

\end{document}





